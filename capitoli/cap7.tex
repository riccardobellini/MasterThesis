\chapter{Conclusioni e sviluppi futuri}
\label{chap:conclusioni}
\vspace{1cm}
L'obiettivo di questo capitolo \`e fornire un rapido riepilogo del lavoro svolto
e descritto in questa tesi, con le motivazioni per cui si \`e ritenuto necessario
svolgere il lavoro, nella sezione \ref{sec:osservazioniConclusiveCap7}. Vengono inoltre
presentate alcune limitazioni di questo lavoro.

La sezione \ref{sec:sviluppiFuturi} descrive cosa pu\`o essere fatto in futuro
per migliorare la metodologia proposta, alla luce delle limitazioni elencate
nella sezione precedente.

\newpage


\section{Osservazioni conclusive}
\label{sec:osservazioniConclusiveCap7}
La costante ricerca di prestazioni sempre migliori per i sistemi informativi
ha dato una forte spinta alla ricerca nell'ambito delle architetture hardware.
Spesso infatti, vi \`e necessit\`a di realizzare architetture dedicate in grado di
velocizzare l'esecuzione di un'applicazione.

In tale processo, le architetture riconfigurabili rappresentano una valida opzione
per ottenere performance elevate, senza sacrificare la flessibilit\`a che non si avrebbe
utilizzando sistemi \ac{ASIC}. Tuttavia, sfruttare in modo efficace le potenzialit\`a
offerte dai sistemi riconfigurabili \`e un compito impegnativo, in particolare a causa
del lungo processo di design di questi sistemi, soggetto a errori, come nel caso dei sistemi
\ac{ASIC}.
Inoltre, i tool di sviluppo esistenti offrono un supporto limitato al designer, e non permettono
di valutare quando utilizzare la riconfigurazione parziale dinamica apporti miglioramenti
all'esecuzione dell'applicazione sul sistema finale.

In questo lavoro \`e stato proposto \emph{MapR}, una metodologia
altamente automatizzata per la progettazione di \acs{MPSoC} riconfigurabili, il cui
obiettivo \`e permettere al designer di valutare l'impatto della riconfigurazione prima che il sistema
sia implementato; ci\`o consente di determinare quando utilizzando la riconfigurazione si pu\`o
ottenere uno speedup rispetto a un'esecuzione su un processore general-purpose oppure
su core statici implementati su logica riconfigurabile.
Per la realizzazione di MapR sono state impiegate euristiche sia per il design dell'architettura, che per
le fasi di mapping e scheduling dell'applicazione da eseguire; la soluzione cos\`i ottenuta \`e
subottimale, tuttavia \`e possibile avere comunque una stima dello speedup ottenibile
facendo uso di riconfigurazioni. La metodologia utilizza una fase di scheduling
per il calcolo della metrica relativa al \mbox{makespan}; 
tale metrica \`e calcolata da uno scheduler list-based che cerchi di sfruttare tecniche quali
\emph{configuration prefetching} per mascherare l'overhead introdotto dalle
riconfigurazioni e che consideri esplicitamente le comunicazioni da eseguire, oltre
alla possibilit\`a di riutilizzare i moduli per ridurre il numero di riconfigurazioni.

Sono stati presentati i risultati relativi all'esecuzione degli algoritmi descritti
nel capitolo \ref{chap:approccio}, che mostrano come la meta-euristica evolutiva basata
su \acl{ACO} permetta di esplorare efficacemente lo spazio delle soluzioni,
per fornire al designer una stima dell'impatto della riconfigurazione sul sistema finale.

Pur permettendo di stimare lo speedup derivante dall'esecuzione di un'applicazione su hardware riconfigurabile,
la metodologia proposta presenta alcune limitazioni. In primo luogo,
la mancanza di una fase di \emph{floorplanning}, che assicuri
l'effettiva fattibilit\`a del piazzamento dei moduli calcolati durante il mapping.
La terza limitazione deriva dal fatto che, una volta calcolate le informazioni di mapping
tramite l'algoritmo di esplorazione, la generazione e connessione dei core \`e lasciata
al designer, che deve creare manualmente l'architettura target.

\section{Sviluppi futuri}
\label{sec:sviluppiFuturi}
Il primo sviluppo futuro consiste nell'estensione della metodologia per considerare una fase di
\emph{floorplanning} che fornisca un feedback per la fattibilit\`a del piazzamento, dato
il mapping calcolato dall'algoritmo. In caso di feedback negativo l'esplorazione pu\`o essere rieseguita
riducendo l'area a disposizione, per calcolare un nuovo mapping; alternativamente, nell'eventualit\`a
di poter disporre di un algoritmo euristico molto veloce per il floorplanning, si pu\`o
controllare la fattibilit\`a del piazzamento di ogni soluzione calcolata, generando in questa maniera
solamente mapping fisicamente piazzabili.

Oltre alla fase di floorplanning, possibili lavori futuri riguardano l'implementazione di
algoritmi per valutare metriche aggiuntive, oltre al makespan, come quelle descritte
nella sezione \ref{subsec:mappingScheduling}:
\begin{itemize}
 \item metrica relativa all'uso dell'area;
 \item metrica relativa a una stima del consumo di energia di un determinato mapping.
\end{itemize}


Infine, per automatizzare ulteriormente il procedimento, sarebbe utile estendere la metodologia
proposta per generare automaticamente l'architettura target definita in fase di esplorazione; invece
che lasciare al designer il compito di creare i core e connetterli manualmente, tale procedimento
potrebbe essere svolto in automatico dal sistema.