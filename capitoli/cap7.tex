\chapter{Conclusioni}
\label{chap:conclusioni}
\vspace{1cm}
In questo lavoro \`e stato proposto \emph{MapR}, una metodologia
altamente automatizzata per la progettazione di \acs{MPSoC} riconfigurabili, che permette
al progettista di valutare l'impatto della riconfigurazione durante la fase di design.
Il lavoro oggetto della tesi combina euristiche sia per il design dell'architettura, che per
le fasi di mapping e scheduling dell'applicazione da eseguire.

Sono stati presentati i risultati relativi all'esecuzione degli algoritmi descritti
nel capitolo \ref{chap:approccio}, che mostrano come la meta-euristica evolutiva basata
su \ac{ACO} permetta di esplorare efficacemente lo spazio delle soluzioni,
per fornire al designer una stima dell'impatto della riconfigurazione sul sistema finale.

In questo lavoro \`e stata presentata soltanto la metrica relativa al makespan,
calcolata da uno scheduler list-based che cerchi di sfruttare tecniche quali
\emph{configuration prefetching} per mascherare l'overhead introdotto dalle
riconfigurazioni e che consideri esplicitamente le comunicazioni da eseguire.

