\chapter{Introduzione}
\label{chap:intro}
\vspace{1cm}
L'obiettivo di questo capitolo è fornire al lettore un'introduzione agli argomenti 
trattati nel corso della tesi, oltre alle motivazioni e a una panoramica del contesto in 
cui si inquadra questo lavoro.

La Sezione \ref{sec:reconfComp} contiene una descrizione dell'area generale in cui si 
svolge il lavoro, presentando i concetti fondamentali alla base del \emph{reconfigurable 
computing}. Viene inoltre fornita una descrizione del progetto di ricerca all'interno del 
quale si colloca il lavoro svolto.

La Sezione \ref{sec:definizioneProblema} descrive quindi la problematica dello 
\emph{scheduling}, oggetto del lavoro, alla luce di quanto descritto nella sezione 
precedente.


\section{Reconfigurable Computing}
\label{sec:reconfComp}
Il continuo progresso tecnologico ha portato a una sempre maggiore riduzione delle 
dimensioni dei transistor e a un aumento crescente del numero di componenti che possono 
essere posizionati su un chip. Conseguentemente, oggigiorno è possibile implementare un 
intero sistema hardware-software 

\subsection{Il progetto \emph{FASTER}}
\label{sec:progettoFaster}


\section{Definizione del problema}
\label{sec:definizioneProblema}

