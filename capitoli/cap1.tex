\chapter{Introduzione}
\label{chap:intro}
\vspace{1cm}
L'obiettivo di questo capitolo è fornire al lettore un'introduzione agli argomenti 
trattati nel corso della tesi, oltre alle motivazioni e a una panoramica del contesto in 
cui si inquadra questo lavoro.

La Sezione \ref{sec:reconfComp} contiene una descrizione dell'area generale in cui si 
svolge il lavoro, presentando i concetti fondamentali alla base del \emph{reconfigurable 
computing}.

La Sezione \ref{sec:definizioneProblema} descrive quindi la problematica dello 
\emph{scheduling}, oggetto del lavoro, alla luce di quanto descritto nella sezione 
precedente.


\section{Reconfigurable computing}
\label{sec:reconfComp}
Oggigiorno, grazie al progresso tecnologico e alla costante riduzione della dimensione 
dei transistor, è resa possibile l'integrazione di diversi componenti elettronici su un 
singolo chip; questi sistemi, chiamati \ac{SoC} o \ac{MPSoC}, a seconda 
che il microprocessore abbia uno o più core, comprendono oltre al processore anche vari 
moduli funzionali quali blocchi di memoria, connettori per interfacce (USB, FireWire, 
Ethernet ecc.) e altre periferiche, collegati tramite BUS.

Questa integrazione tuttavia apre una serie di problemi, tra cui elevato costo di 
progettazione dei sistemi, bassa affidabilità e necessità di controllare il consumo di 
energia; i problemi sopra citati possono essere risolti grazie all'impiego del 
reconfigurable computing.

Con il termine \emph{reconfigurable computing} si intende un'architettura hardware che 
offre la possibilità di essere riconfigurata per implementare qualsiasi logica l'utente 
desideri. Le caratteristiche e il fondamento logico alla base di questo tipo di 
architetture sono descritte nelle prossime sezioni.

\subsection{Da Von Neumann alle architetture riconfigurabili}
\label{subsec:cambioParadigma}
L'obiettivo delle architetture riconfigurabili consiste nel combinare i 
vantaggi dei due paradigmi di computazione principali: \emph{general-purpose computing} e 
\emph{application-specific computing}. Vengono ora descritti brevemente i due paradigmi 
principali e le differenze rispetto alle architetture riconfigurabili, per meglio 
evidenziare i vantaggi derivanti dall'utilizzo di un'architettura riconfigurabile.


\section{Definizione del problema}
\label{sec:definizioneProblema}

