\chapter{Introduzione}
\label{chap:intro}
\vspace{1cm}
L'obiettivo di questo capitolo è fornire al lettore un'introduzione agli argomenti 
trattati nel corso della tesi, oltre alle motivazioni e a una panoramica del contesto in 
cui si inquadra questo lavoro.

La Sezione \ref{sec:reconfComp} contiene una descrizione dell'area generale in cui si 
svolge il lavoro, presentando i concetti fondamentali alla base del \emph{reconfigurable 
computing}.

La Sezione \ref{sec:definizioneProblema} descrive quindi la problematica dello 
\emph{scheduling}, oggetto del lavoro, alla luce di quanto descritto nella sezione 
precedente.


\section{Reconfigurable Computing}
\label{sec:reconfComp}
Oggigiorno, grazie al progresso tecnologico e alla costante riduzione della dimensione 
dei transistor, si assiste all'integrazione di diversi componenti elettronici su un 
singolo chip. Questi sistemi, chiamati \ac{SoC} o \ac{MPSoC}, a seconda che il 
microprocessore abbia uno o più core, comprendono oltre al processore anche vari moduli 
funzionali quali blocchi di memoria, connettori per interfacce (USB, FireWire, Ethernet 
ecc.) e altre periferiche, collegati tramite BUS.


\section{Definizione del problema}
\label{sec:definizioneProblema}

