\chapter{Stato dell'arte}
\label{chap:SOA}
\vspace{1cm}
In questo capitolo verrà fornito un esame dello stato dell'arte e verranno presentate 
alcune soluzioni relative al lavoro oggetto di questa tesi.

Nella Sezione \ref{sec:progettoFASTER} è descritto il progetto europeo 
\ac{FASTER} con una spiegazione sulla metodologia proposta per la toolchain e 
l'algoritmo che effettua l'analisi esplorativa delle soluzioni, che interagisce 
con lo scheduler.

Nella Sezione \ref{sec:algoritmiProposti} si esaminano i vari approcci proposti 
dagli autori nell'ambito del problema dello scheduling dei task su dispositivi 
riconfigurabili, divisi in algoritmi esatti, trattati nella Sezione 
\ref{sec:algoritmiEsatti}, ed euristici, trattati nella Sezione 
\ref{sec:algoritmiEuristici}.


\section[Il progetto \acs{FASTER}]{Il progetto \acs{FASTER}}
\label{sec:progettoFASTER}
In questa sezione vengono presentate il fondamento logico e le motivazioni alla 
base del progetto \ac{FASTER}. La Sezione \ref{sec:fasterIntro} fornisce 
un'introduzione al progetto e ne spiega gli obiettivi; la Sezione 
\ref{sec:fasterMotivazioni} delinea le motivazioni alla base che hanno 
portato alla creazione del progetto; infine, la Sezione 
\ref{sec:fasterMetodologia} illustra la metodologia tramite la quale si possono 
raggiungere gli obiettivi prefissati.

\subsection{Introduzione}
\label{sec:fasterIntro}
\ac{FASTER} è l'acronimo di \emph{\acl{FASTER}}, un progetto\footnote{Sito del 
progetto: \url{http://www.fp7-faster.eu/}.} organizzato dalla comunità europea 
che coinvolge diverse aziende e atenei\footnote{I collaboratori del progetto 
sono: l'\ac{ICS} di \ac{FORTH} in Grecia, Chalmers University of Technology in 
Svezia, l'università di Ghent in Belgio, l'Imperial College di Londra, il 
Politecnico di Milano e le aziende Maxeler, ST Microelectronics e Synelixis.}, 
tra cui il Politecnico di Milano. Il progetto è stato avviato il 1 settembre 
2011, ha una durata di 36 mesi ed è supportato tramite fondi stanziati dalla 
comunità europea.


\subsection{Motivazioni}
\label{sec:fasterMotivazioni}
% Extending product functionality and lifetime requires constant addition of new 
% features to satisfy the growing customer needs and the evolving market and 
% technology trends. Software component adaptivity is straightforward but not 
% enough: recent products include hardware accelerators - for reasons of 
% performance and power efficiency - that also need to adapt to new 
% requirements.For example, a Network Intrusion Detection System (NIDS) needs to 
% scan all incoming network packets for suspicious content. The scanning has to be 
% fast so that the monitored communication links are not slowed down, while the 
% list of threats to check for is extended and updated on a daily basis.
% 
% Reconfigurable logic allows the definition of new functions to be implemented in 
% dynamically instantiated hardware units, combining adaptivity with hardware 
% speed and efficiency. For the Intrusion Detection System example, new rules can 
% be hardcoded into the reconfigurable logic, achieving high performance, while 
% providing the necessary adaptivity to new threats.
% 
% FASTER will facilitate the use of reconfigurable technology by providing a 
% complete methodology that enables designers to easily implement and verify 
% applications on platforms with general-purpose processors and acceleration 
% modules implemented in the latest reconfigurable technology. We expect that the 
% project will lead to a 20% productivity improvement due to seamless 
% implementation and verification of dynamically changing systems, a 50% total 
% ownership cost reduction for NIDS and Reverse Time Migration systems, with a 2x 
% performance improvement under power constraints for Global Illumination and 
% Image Analysis. 


\subsection{Metodologia utilizzata}
\label{sec:fasterMetodologia}

\section{Algoritmi proposti}
\label{sec:algoritmiProposti}


\subsection{Algoritmi esatti}
\label{sec:algoritmiEsatti}


\subsection{Algoritmi euristici}
\label{sec:algoritmiEuristici}


% TODO presentazione delle motivazioni che hanno portato allo sviluppo di un 
% algoritmo iterativo basato su una lista e non esplorativo

% Il flusso di esecuzione della toolchain prevede l'invocazione del tool di 
% scheduling statico come componente esterno da parte dell'algoritmo di 
% esplorazione dello spazio di design del sistema, per la valutazione di una 
% particolare metrica, la stima del tempo di esecuzione totale dello schedule 
% dato un determinato mapping. Poichè la fase di mapping è implementata sotto 
% forma di un  