\chapter{Stato dell'arte}
\label{chap:SOA}
\vspace{1cm}
In questo capitolo verrà fornito un esame dello stato dell'arte e verranno presentate 
alcune soluzioni relative al lavoro oggetto di questa tesi.

Nella Sezione \ref{sec:progettoFASTER} è descritto il progetto europeo 
\ac{FASTER} con una spiegazione sulla metodologia proposta per la toolchain e 
l'algoritmo che effettua l'analisi esplorativa delle soluzioni, che interagisce 
con lo scheduler.

Nella Sezione \ref{sec:algoritmiProposti} si esaminano i vari approcci proposti 
dagli autori nell'ambito del problema dello scheduling dei task su dispositivi 
riconfigurabili, divisi in algoritmi esatti, trattati nella Sezione 
\ref{sec:algoritmiEsatti}, ed euristici, trattati nella Sezione 
\ref{sec:algoritmiEuristici}.


\section[Il progetto \acs{FASTER}]{Il progetto \acs{FASTER}}
\label{sec:progettoFASTER}


\section{Algoritmi proposti}
\label{sec:algoritmiProposti}


\subsection{Algoritmi esatti}
\label{sec:algoritmiEsatti}


\subsection{Algoritmi euristici}
\label{sec:algoritmiEuristici}


% TODO presentazione delle motivazioni che hanno portato allo sviluppo di un 
% algoritmo iterativo basato su una lista e non esplorativo

% Il flusso di esecuzione della toolchain prevede l'invocazione del tool di 
% scheduling statico come componente esterno da parte dell'algoritmo di 
% esplorazione dello spazio di design del sistema, per la valutazione di una 
% particolare metrica, la stima del tempo di esecuzione totale dello schedule 
% dato un determinato mapping. Poichè la fase di mapping è implementata sotto 
% forma di un  