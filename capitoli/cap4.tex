\chapter{Implementazione proposta}
\label{chap:implementazione}
\vspace{1cm}
Il Capitolo \ref{chap:approccio} è stato interamente dedicato alla descrizione 
dell'approccio adottato nel lavoro oggetto di questa tesi, al fine di 
realizzare un baseline scheduler che tenga conto di possibili 
\emph{riconfigurazioni} e \emph{comunicazioni}, capace di fornire una 
valutazione quantitativa della bontà di un dato mapping.

Questo capitolo è dedicato alla descrizione dei dettagli implementativi 
riguardanti la realizzazione dell'algoritmo di scheduling.

Il capitolo è strutturato nel seguente modo: la Sezione \ref{sec:struttureDati} 
contiene la descrizione delle strutture dati relative all'implementazione dello 
scheduler; la Sezione \ref{sec:classiGerarchie} descrive le classi e le 
gerarchie che compongono la parte della toolchain relativa alla parte di 
mapping/scheduling (Sezione \ref{subsec:strutturaToolchain}) e i componenti 
dell'algoritmo di scheduling nel dettaglio (Sezione 
\ref{subsec:componentiScheduler}).


\section{Strutture dati}
\label{sec:struttureDati}


\section{Classi e gerarchie principali}
\label{sec:classiGerarchie}


\subsection{Struttura toolchain}
\label{subsec:strutturaToolchain}


\subsection{Componenti dello scheduler}
\label{subsec:componentiScheduler}

