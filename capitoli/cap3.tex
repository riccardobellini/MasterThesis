\chapter{Approccio proposto}
\label{chap:approccio}
\vspace{1cm}
Nel Capitolo \ref{chap:SOA} sono state descritte alcune soluzioni proposte 
relative al lavoro oggetto di questa tesi, in particolare algoritmi esatti ed
euristici di ottimizzazione basati su tecniche di evoluzione.
% TODO & FIXME aggiungere qualcosa

In questo capitolo verrà descritto l'approccio utilizzato per risolvere il 
problema di scheduling dei task considerando comunicazioni e riconfigurazioni.

Il capitolo è organizzato secondo la seguente struttura: nella Sezione 
\ref{sec:panoramicaMetodologia} viene descritto ad alto livello il 
funzionamento dell'algoritmo realizzato, viene fornita una panoramica delle 
fasi in cui questo si divide, con particolare enfasi sulle interfacce esterne 
che collegano il componente agli altri strumenti utilizzati nella toolchain di 
\acs{FASTER}; la Sezione \ref{sec:euristicaSceltaTask} contiene una descrizione 
dettagliata della fase più delicata del procedimento di scheduling, la scelta 
del task migliore da considerare ad ogni passo di decisione; la Sezione 
\ref{sec:osservazioniConclusive} fornisce un riepilogo dei concetti importanti 
presentati in questo capitolo.


\section{Panoramica della metodologia}
\label{sec:panoramicaMetodologia}


\section{Euristica di scelta dei task}
\label{sec:euristicaSceltaTask}


\section{Osservazioni conclusive}
\label{sec:osservazioniConclusive}
