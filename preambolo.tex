\usepackage[italian,english]{babel}
\selectlanguage{english}
\hyphenation{task}
\selectlanguage{italian}
\hyphenation{task}
\usepackage[printonlyused]{acronym}
\usepackage[footnotesize]{caption}
\usepackage[utf8]{inputenc}
\usepackage{float}
\usepackage{graphicx}
\usepackage{indentfirst}
\usepackage{longtable}
\usepackage{pdfpages}
\usepackage{tikz,fullpage}
\usetikzlibrary{arrows}
\usepackage{textcomp}
\usepackage{amsfonts, amsmath, amsthm, amssymb, epsfig, fancyvrb}
\usepackage{hyphenat}
\usepackage{listings, booktabs}
\usepackage[titletoc]{appendix}

\usepackage{hyperref}
\hypersetup{
breaklinks=true,
pdfborder={0 0 0},
pdfstartview={FitH},
pdfpagemode={UseOutlines},
pdftitle={SMASH Scheduler Reconfiguration and communication-aware scheduling 
of directed acyclic task graphs on reconfigurable devices}, % FIXME
pdfauthor={Riccardo Bellini},
pdfsubject={},
pdfkeywords={}
}


\renewcommand*{\lstlistingname}{Listato}
\renewcommand*{\lstlistlistingname}{Elenco dei listati}
\renewcommand*{\appendixpagename}{Appendici}

\newenvironment{code}{\captionsetup{aboveskip=-0.5em}\begin{listing}[H]}{\end{listing}}

\renewcommand{\rmdefault}{ppl}
\usepackage[scaled]{helvet}
\usepackage{courier}
\normalfont

\frenchspacing
\linespread{1.5}

\newtheorem{mydef}{Definizione}

\newcommand{\eg}{\textit{e.g.,}~}
\newcommand{\ie}{\textit{i.e.,}~}