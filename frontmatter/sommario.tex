\chapter*{Sommario}
\selectlanguage{italian}
\vspace{0.5cm}
Sfruttare efficacemente le potenzialit\`a offerte dall'utilizzo di hardware riconfigurabile
\`e tuttora un compito impegnativo, a causa del lungo processo di design di questi sistemi,
con un alto rischio di errori: la mancanza di metodologie che facilitino il design e
la difficolt\`a di queste ad adattarsi all'applicazione da eseguire costituiscono
un ostacolo all'adozione di questa tecnologia.
Inoltre, gli approcci esistenti non sono in grado di sfruttare in maniera efficace
l'opportunit\`a di utilizzare la \emph{riconfigurazione parziale dinamica}, in inglese
\ac{DPR} sin dalle prime fasi del processo di design, ottenendo soluzioni subottimali.

In questo lavoro viene proposto \emph{MapR}, una metodologia
altamente automatizzata per la progettazione di \acs{MPSoC} riconfigurabili, che permette
al progettista di valutare l'impatto della riconfigurazione durante la fase di design.
Il lavoro oggetto della tesi combina euristiche sia per il design dell'architettura, che per
le fasi di mapping e scheduling dell'applicazione da eseguire.
Nello specifico:
\begin{itemize}
  \item l'algoritmo di mapping \`e costituito da una meta-euristica evolvibile basata su un
    algoritmo meta-euristico noto come \emph{\ac{ACO}}, che permette di esplorare lo spazio delle
    soluzioni possibili effettuando ottimizzazione multiobiettivo;
  \item l'algoritmo di scheduling fornisce una valutazione quantitativa del mapping calcolato,
    fornendo cos\`i una possibile metrica per l'ottimizzazione.
\end{itemize}
L'algoritmo di scheduling del framework \`e esteso per tenere in considerazione
le riconfigurazioni e mascherarle automaticamente quando possibile, per migliorare le
performance complessive.
% TODO sommario dei risultati

