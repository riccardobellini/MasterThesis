\chapter*{Sommario}
\selectlanguage{italian}
\vspace{0.5cm}
Sfruttare efficacemente le potenzialit\`a offerte dall'utilizzo di hardware riconfigurabile
\`e tuttora un compito impegnativo: la mancanza di metodologie adatte e
la difficolt\`a di queste ad adattarsi all'applicazione da eseguire costituiscono
un'ostacolo all'adozione di questa tecnologia.

In questo lavoro viene proposto \emph{PaRA-Sched}, un miglioramento di una metodologia
altamente automatizzata per la progettazione di \acs{MPSoC} riconfigurabili, che permette
al progettista di valutare l'impatto della riconfigurazione durante la fase di design.
Nello specifico, l'algoritmo di scheduling del framework \`e esteso per tenere in considerazione
le riconfigurazioni e mascherarle automaticamente quando possibile, per migliorare le
performance complessive.
% TODO sommario dei risultati

