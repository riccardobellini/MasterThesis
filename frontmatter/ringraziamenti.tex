\chapter*{Ringraziamenti}
Eccoci qua, alla fine di questo percorso universitario e a pochi giorni dalla consegna
definitiva della tesi, finalmente si ripensa a tutto quello che si \`e passato
in questi anni e, per l'ennesima volta, si ricordano le persone che hanno reso possibile tutto
questo grazie al loro sostegno.
Questa tesi non sarebbe mai stata scritta senza la collaborazione diretta di alcune
persone, ma senza tante altre persone difficilmente sarei arrivato al punto
di scrivere queste parole.

Innanzitutto non posso non cominciare ringraziando i miei genitori, Elena e Luigi,
non sarei mai arrivato in fondo a questo percorso di studi senza il loro supporto
morale ed economico.

Ringrazio Micaela per aver sempre creduto che ce la potessi fare, anche quando
tutto mi sembrava nero, per avermi sempre predetto il passaggio degli esami
quando io stesso non ci credevo (avevi ragione anche in quei momenti,
tanto per cambiare! \verb+:p+) e per aver continuato a sopportarmi durante il periodo
di tesi.

Ringrazio i miei parenti, i miei nonni Giovanna (Giannina) e Renato,
i miei zii Pietro Carlo (Caio), Massimo, Luigi, Patrizia e cugini Marco, Chiara;
ringrazio coloro che c'erano quando ho iniziato questo viaggio e che adesso
non possono essere qui a festeggiare, nonno Francesco e zia Carla, mi mancate.

Ringrazio i miei colleghi e amici Andrea, Enrico, Giovanni e Luca, per le giornate passate
a studiare assieme cercando di capire come si risolvessero gli esercizi,
per il \emph{reverse engineering} e il \emph{sistema a cavolate convergenti},\footnote{Il
termine ``cavolate'' \`e stato volontariamente edulcorato onde preservare
la gi\`a povera decenza lessicale di questa tesi.} per le risate che ci siamo
fatti nelle varie occasioni; il Poli \`e stato senza dubbio pi\`u fattibile
grazie a voi.

Infine, ringrazio le persone che pi\`u materialmente hanno contribuito
alla stesura di questa tesi; Marco Domenico Santambrogio, per essere stato
mio relatore e avermi dato l'opportunit\`a di passare qualche giorno a
Boston, cos\`i da conoscere diverse realt\`a d'oltreoceano e da avere un piccolo
``assaggio'' di America. Ringrazio Riccardo Cattaneo e Gianluca Durelli, preziosi e validi
alleati in fatto di correzioni e di meeting strategici per come andare avanti con
il lavoro. Ringrazio Christian Pilato, per i consigli che mi ha fornito e Alessandro
Nacci, per l'aiuto nella realizzazione della piccolissima parte dell'interfaccia
grafica a cui ho contribuito.

Ringrazio i ragazzi e le ragazze del NECST Lab, in particolare Marco Rabozzi
per il validissimo sostegno sulla parte di ricerca operativa e formulazioni
\acs{ILP}; il buon Mazzu,\footnote{Oooooh :).} John (grazie per
il ``Keep Calm and Get Graduate''!), Mario, Ale, Ezio, Davide e tutti gli altri per le partite
a \emph{bam-bam}, le risate e le trollate. In generale, grazie per le esperienze
che abbiamo condiviso, hanno contribuito a farmi crescere dal punto di vista del
lavoro di gruppo.

\vspace{1.5em}
\begin{flushright}
 \textit{Riccardo}
\end{flushright}
