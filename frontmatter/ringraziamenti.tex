\chapter*{Ringraziamenti}
È difficile iniziare a scrivere, è frustrante guardare il cursore lampeggiare 
su una pagina quasi vuota, implacabilmente in attesa che qualche frase sensata 
venga scritta. Molto probabilmente queste frasi non hanno senso, tuttavia 
voglio iniziare così. È ancora più difficile iniziare a scrivere dei 
ringraziamenti, si rischia di avere paura che qualcuno si senta sminuito perchè 
non sono state spese le giuste parole, di non sapersi esprimere oppure di 
tralasciare qualcuno. E qualcuno viene inevitabilmente dimenticato, d'altronde 
come si fa a condensare in poche righe e in pochi nomi tutte le persone che ti 
hanno sostenuto, direttamente o indirettamente, nel corso di tutti questi anni? 
Questa tesi non sarebbe mai stata scritta senza la collaborazione diretta di alcune 
persone, ma senza tante altre persone io difficilmente sarei arrivato al punto 
di scrivere queste parole, per cui non me ne abbiano a male se non vengono 
direttamente menzionati, a loro sono grato con tutto il cuore.

Innanzitutto non posso non cominciare ringraziando i miei genitori e tutti i 
miei parenti, non sarei mai arrivato in fondo a questo percorso di studi senza 
il loro supporto morale ed economico.

Ringrazio Micaela per aver sempre creduto che ce la potessi fare, in ogni 
occasione, avermi sempre predetto il passaggio degli esami quando io stesso 
non ci credevo (avevi ragione anche in quei momenti, tanto per cambiare!) e, 
ultimo ma non meno importante, per aver continuato a sopportare un ingegnere!

Ringrazio i miei colleghi e amici Andrea, Enrico e Giovanni, per le giornate 
passate a studiare assieme cercando di capire come si risolvessero gli 
esercizi, per il reverse engineering, il ``sistema a cavolate convergenti'' (il 
termine cavolate è stato volontariamente edulcorato, n.d.r.) e per le risate 
che ci siamo fatti nelle varie occasioni.

Infine, ringrazio le persone che più materialmente hanno contribuito alla 
stesura di questa tesi: Marco Domenico Santambrogio, per essere stato mio 
relatore e avermi dato l'opportunità di passare qualche giorno a Boston, così 
da conoscere diverse realtà d'oltreoceano; Riccardo Cattaneo, Gianluca 
Durelli e Christian Pilato, per i consigli che mi hanno fornito e per il 
tempo che hanno dedicato a seguire il lavoro e accertarsi che tutto filasse 
liscio; Alessandro Nacci per l'aiuto nella realizzazione della piccolissima 
parte di interfaccia grafica a cui ho contribuito, pur non essendo 
strettamente oggetto del lavoro di questa tesi merita un ringraziamento. 
Ringrazio anche i ragazzi e le ragazze del NECST Lab per le esperienze che 
abbiamo condiviso, che hanno sicuramente contribuito a farmi crescere dal punto 
di vista del lavoro di gruppo e dello stare assieme divertendosi. Senza queste 
cose sarebbe stato molto più difficile arrivare fino in fondo.

\vspace{1.5em}
\begin{flushright}
 \textit{Riccardo}
\end{flushright}
